% ---- Don't modify from Line no. 2 to 74 ----
\documentclass[12pt]{article}

\usepackage{lineno,hyperref}
\modulolinenumbers[5]
\usepackage{graphics}
\usepackage{graphicx}
\usepackage{cite}
\usepackage{epsfig}
\usepackage{amsmath}   
\usepackage{amssymb}
\usepackage{placeins}
\usepackage[linesnumbered,ruled,vlined]{algorithm2e}
\usepackage{setspace}
\usepackage{multirow}
\usepackage[export]{adjustbox}[2011/08/13]
\usepackage{tabularx}
\usepackage{algcompatible}
\usepackage{caption}
\usepackage{epsf}
\usepackage{epstopdf}
\usepackage{subfigure} 
\usepackage{colortbl}
\usepackage{longtable}
\usepackage{enumerate}
\usepackage{tabularx, booktabs}

\usepackage[table,xcdraw]{xcolor}

\usepackage{tikz}
\usepackage{multirow}
\usepackage{enumitem}
\usepackage{soul}
\usepackage{xcolor}
\usepackage[utf8]{inputenc}
\usepackage{placeins}
\usepackage{makecell}
\newcounter{qcounter}
\usepackage{tcolorbox}
\usepackage{lscape}
\usepackage{url}
\usepackage{hyperref}
\usepackage{tablefootnote}
\usepackage{url}
\usepackage{geometry}
 \geometry{
 a4paper,
 total={170mm,257mm},
 left=20mm,
 top=20mm,
 }

\usepackage{hyperref}
\hypersetup{
    colorlinks=true,
    linkcolor=blue,
    filecolor=magenta,      
    urlcolor=cyan,
}

\setlength{\parindent}{4em}
\setlength{\parskip}{1em}
\renewcommand{\baselinestretch}{1.5}

\usepackage[numbers]{natbib}
\bibliographystyle{unsrtnat}

\begin{document}

% ------------ Don't modify anything up to here ---------------

% From here on-wards modify only the relevant fields, such as Title (line no. 76), "Section:", "Course Instructor:", and "Team Members:" field. (Team Members details should be in the format such as, name, reg. no., mobile no. and email id.). Further, "Title:" can be changed as per your selected topic name. In the brief description field you can describe your topic in 250 words. Additionally, in the "Key Feature:" field add your mini-project feature as an item (example is shown). AT the end in the "Reference:" field add the website/paper/article referred for this mini-project as an item.

\begin{center}
    \textbf{\Large{Mini Project Proposal \\
    (\textcolor{blue}{FTMS-Farmers' Transaction Management System})}}
\end{center}

\noindent 
\textbf{Course Code:} CS251 
\hspace{2in} 
\textbf{Course Title:} Database Systems(Minor) \\
\textbf{Semester:} B. Tech 4$^{th}$ Sem 
\hspace{1.6in} 
\textbf{Academic Year:} 2020-21 \\
\hspace{1.8in} 
\textbf{Team Members:} \\
\textbf{1.} R Raghavendra, 191CH036, 98802 34074, rraghavendra.191ch036@nitk.edu.in
\newline
\textbf{2.} Pranshu Shukla, 191ME260, 73859 25943, pranshushukla.191me260@nitk.edu.in
\newline
\textbf{3.} Yuvasankar B, 191ME197,  80951 77040, yuvasankarb.191me197@nitk.edu.in
\newline
\textbf{4.} Sankarsh R, 191EC261, 91487 36087, sankarsh.191me275@nitk.edu.in

\vspace{0.25in}

\noindent
\textbf{Brief Description:}
\newline
For a farmer getting crops harvested after many months' long process is just half of the task, getting crops sold to market-place can be tiring, hard and extremely disadvantageous to a farmer if he is not completely acquainted with the process. This is where the FTMS steps in; it is a simple all-in-one application for farmers, retailers(buyers) and corporations alike that enables an easy flow of information and communication between the users. This application allows retailers to raise quotes on purchase of crops from farmers which not only makes it easier for farmers to contact their buyers directly but also allows other customers to see their competitions. This process removes the need for a middleman for the purchasing process which tend to take their own proportions of money from farmers. All a farmer has to do is choose retailers from a list of retailers who are offering the best quote for the crop and contact him directly instead of scrambling in a market searching and asking around for the best buyer. With just a click away, all these features can really come to aid for anyone associated with the agriculture business. But mainly, from this project, we hope to create an easier world for farmers who are the driving force of our nation's growth. 

\noindent
\textbf{Key Features:}
\begin{enumerate}
    \item Login and Sign-Up facility.
    %\item Separate portals for farmers and buyers.
    \item Buyer search by User-ID
    \item Best search based on quote for farmers
    \item Best search based on location for farmers
    \item Adding and removing quote feature for buyers
    \item Modifying quotes feature for buyers
\end{enumerate}


\noindent
\textbf{Tables:}
\begin{enumerate}
    \item Farmer - Details of farmers registered.
    \item Buyer - Details of buyers registered.
    \item Crop - Crops handled by the application, with crop IDs.
    \item Crop\textunderscore Grown - Table for mapping farmers to the crops they grow
    \item Transactions - Table showing the details of transactions between farmers and buyers till date.
    \item Quotations - Table of currently available quotations from buyers.
\end{enumerate}

\noindent
\bold{NOTE: The above description is tentative and describes only the outline of the project, there may be changes in the final project.}

\noindent
\textbf{References:}
\begin{enumerate}
    \item https://farmityourself.com/how-do-small-farmers-sell-their-crops/
    \item https://www.farmerslink.org.uk/where-and-how-farmers-can-sell-their-produce/
    \item https://www.thebetterindia.com/101983/farmer-friend-online-vegetables-direct-from-farmers/
\end{enumerate}



\begin{center}
    \textbf{**** END ****}
\end{center}

\end{document}
